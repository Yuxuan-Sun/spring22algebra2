\documentclass[a4paper]{article}

\usepackage[utf8]{inputenc}
\usepackage{amssymb, amsfonts, latexsym, amsthm, amsmath, framed}
\usepackage{esvect}
\usepackage{parskip}

\usepackage{amsmath, amssymb, framed, tcolorbox}
\tcbuselibrary{theorems}

\usepackage{graphicx}

\usepackage{mathrsfs}
\usepackage[hidelinks,colorlinks=true,linkcolor=blue,citecolor=blue]{hyperref}

\usepackage{xcolor}

\usepackage{natbib}
\bibliographystyle{abbrvnat}
\setcitestyle{authoryear,open={},close={}}

\newcommand{\ba}{\backslash}
\newcommand{\Q}{\mathbb{Q}}
\newcommand{\C}{\mathbb{C}}
\newcommand{\R}{\mathbb{R}}
\newcommand{\N}{\mathbb{N}}
\newcommand{\Z}{\mathbb{Z}}
\newcommand{\F}{\mathbb{F}}
\newcommand{\rank}{\text{rank}}

% figure support
\usepackage{import}
\usepackage{xifthen}
\pdfminorversion=7
\usepackage{pdfpages}
\usepackage{transparent}
\newcommand{\incfig}[1]{%
	\def\svgwidth{\columnwidth}
	\import{./figures/}{#1.pdf_tex}
}

\newtheoremstyle{bfnote}%
  {}{}
  {}{}
  {\bfseries}{.}
  { }{\thmname{#1}\thmnumber{ #2}\thmnote{ (#3)}}

\theoremstyle{bfnote} % set the style for the following theorems

\newtheorem{thm}{Theorem}[section] %\newtheorem{name}{display-text}[numbered-within]

\newtheorem{lem}[thm]{Lemma} %\newtheorem{name}[numbered-like]{display-text}
\newtheorem{cor}[thm]{Corollary}
\newtheorem{prop}[thm]{Proposition}
\newtheorem{alg}[thm]{Algorithm}

\theoremstyle{bfnote}                  % switch to a different style
\newtheorem{defn}[thm]{Definition}
\newtheorem{conj}[thm]{Conjecture}


\theoremstyle{example}                       % another style
\newtheorem{prob}[thm]{Problem}


\theoremstyle{remark}                       % another style
\newtheorem{exmp}[thm]{Example}  % (note:the "example" style is not really good for long examples-- typesets them in italics!)
\newtheorem{rem}[thm]{Remark}
\newtheorem{claim}[thm]{Claim}
\renewcommand{\theclaim}{}

%% If you use numbered equations in a long document, it is preferred to number
% as (x.y), where x is section number, y is equation number

\numberwithin{equation}{section}




\pdfsuppresswarningpagegroup=1

\usepackage{tcolorbox}
\usepackage{tikz}
\tcbuselibrary{theorems, skins, breakable}

\title{Algebra 2 Draft}
\author{Yuxuan Sun}


\begin{document}

\maketitle

\section{Introduction}
This paper aims to introduce the properties of irreducible representations of the symmetric group, an extension of the properties of irreducible representation in a boarder context: topological context, so that hopefully some geometric intuitions could be grasped. Besides abstract algebra, some knowledge about point-set topology is assumed.

The goal is to glance at the infinite-dimensional representations by studying Verma modules.

\section{Lie Algebra}


\begin{defn}[bracket and Lie algebra, \citep{hall}]\label{bracket}
	A vector space $\mathfrak{g}$ over a field $F$, with an operation $\left[ \cdot, \cdot \right] $ from $\mathfrak{g} \times \mathfrak{g} \to  \mathfrak{g}$, called the  \textbf{bracket} or \textbf{commutator} of $x$ and  $y$, is called a  \textbf{Lie algebra} over $F$ if the following properties are satisfied:
\begin{enumerate}
		 \item The bracket operation is bilinear
\begin{enumerate}
	\item $[x,y_1+y_2] = [x,y_1] + [x,y_2]$ and $[x_1+x_2,y] = [x_1,y] + [x_2,y]$
	\item $[\lambda x, y] = \lambda[x,y] = [x, \lambda y]$
\end{enumerate}
		 \item $[x,x] = 0$
		 \item $[x,[y,z]]+[y,[z,x]] +[z,[x,y]] = 0$
	\end{enumerate}
\end{defn}

\bigskip

\begin{thm}[anti-commutativity of bracket] 
	Given a Lie algebra $\mathfrak{g}$ equipped with a bracket operation,for all  $x,y \in \mathfrak{g}$, we always have $[x,y] = -[y,x]$, indicating the \textbf{anti-commutativity of bracket} on $\mathfrak{g}$. 	
\end{thm}

\textit{proof.}
Given any $x,y \in \mathfrak{g}$, using the properties we defined above, we could have:
\begin{align*}
	[x+y, x+y] &= [x, x+y] + [y, x+y] & \textbf{property 1a} \\
		   &= [x,x] + [x,y] + [y,x] + [y,y] & \textbf{property 1a} \\
		   &= [x,y] + [y,x] & \textbf{property 2} \\
	[x,y] &= -[y,x]
\end{align*}

\begin{defn}[adjoint representation of $\mathfrak{sl}_3\C$]\label{adjoint}
	Give $X \in \mathfrak{sl}_3\C$, define a linear map $\operatorname{ad}_X : \mathfrak{sl}_3\C \to \mathfrak{sl}_3\C$ as: \[
		\operatorname{ad}_X(Y) = [X,Y]
	\] the map $X \mapsto \operatorname{ad}_X$ is called an \textbf{adjoint representation} of $\mathfrak{sl}_3\C$.
\end{defn}

\section{Representations of $\mathfrak{sl}_3\C$}
In this section, we are going to look at the adjoint representation of $\mathfrak{sl}_3\C$ and see how it could be decomposed into eigenspaces of a special subspace of  $\mathfrak{sl}_3\C$. This is a special case of a more general statement which would be mentioned at the end of the paper.

Before everything, let's look at what  $\mathfrak{sl}_3\C$ is.

\bigskip

\begin{exmp}[$\mathfrak{sl}_3\C$, \citep{hall}]
	The Lie algebra $\mathfrak{sl}_3\C$ is a vector space of  $3 \times 3$ matrices with trace  $0$ over  $\C$, in other words \[
		\mathfrak{sl}_{3}\C = \left\{ \begin{bmatrix} a & c & d \\
		e & -a+b & f \\ g & h & -b\end{bmatrix} \mid a,b,c,d,e,f,g,h \in \C \right\} 
	\]
	We could see that $\mathfrak{sl}_3\C$ is an  $8$-dimensional vector space with the following basis: \[
	\mathcal{B}_{\mathfrak{sl}_{3}} = \left\{ H_1 = \begin{bmatrix} 1 & 0 &0 \\ 0 & -1 & 0 \\ 0 &0 &0 \end{bmatrix}, H_2 = \begin{bmatrix} 0 &0 &0 \\ 0 &1&0 \\ 0&0&-1 \end{bmatrix}   \right\} \bigcup \left\{ E_{i,j} \mid i \neq j, 1\le i,j \le 3\right\}. 
\] $E_{i,j}$ is a matrix where ${E_{i,j}}_{ij} = 1$ and other places are $0$, for example:\[
		E_{1,3} = \begin{bmatrix} 0 & 0 & 1 \\ 0 & 0 &0 \\ 0 &0 &0 \end{bmatrix}. 
	\] 
\end{exmp}

\begin{rem}[$\mathfrak{h} \subset \mathfrak{sl}_{3}\C$]
	Denote $\mathfrak{h}$ as the two dimenstional subspace of all diagonal matrices in  $\mathfrak{sl}_3\C$ (trace is $0$). Namely:  \[
		\mathfrak{h} = \operatorname{Span}\left\{ H_1=\begin{bmatrix} 1 & 0 &0 \\ 0 & -1 & 0 \\ 0 &0 &0 \end{bmatrix}, H_2 = \begin{bmatrix} 0 &0 &0 \\ 0 &1&0\\0&0&-1 \end{bmatrix}   \right\} 
	\] 
\end{rem}

\bigskip

Now we've chosen our special subspace $\mathfrak{h} \subset \mathfrak{sl}_3\C$, let's look at what it means to be an eigenvector, eigenvalue, and eigenspace for $\mathfrak{h}$.

\begin{defn}[linear functional]
	A linear functional $T$ on a complex vector space  $V$ is a function  $T: V \to \C$ which satisfies the following properties:
	\begin{enumerate}
		\item $T(v+w) = T(v) + T(w)$ 
		\item $T(\alpha v) = \alpha T(v)$
	\end{enumerate}
\end{defn}


\begin{defn}[eigenvector and eigenvalue for a vector space]
	Let $\mathfrak{h}$ be a subspace of a vector space $V$, an  \textbf{eigenvector} for $\mathfrak{h}$ refers to a vector  $v$ that is an eigenvector for all  $H \in \mathfrak{h}$, in other words: \[
		H v = \alpha(H) v \quad \text{for all $H \in \mathfrak{h}$},
	\]  where $\alpha$ is a linear functional on  $H$, and we call $\alpha$ as an \textbf{eigenvalue}. 
\end{defn}

 \begin{defn}[eigenspace for the adjoint action of $\mathfrak{h}$ on  $\mathfrak{sl}_3\C$]
	 Give $\mathfrak{h} \subset \mathfrak{sl}_3\C$ as we defined above, an eigenspace of $\mathfrak{h}$ with a linear functionl  $\alpha$ as an eigenvalue, denoted as  $\mathfrak{g}_{\alpha}$, contains all $M \in \mathfrak{sl}_3\C$ s.t. \[
		 \left[ H,M \right] = \operatorname{ad}_{H}(M) = HM - MH = \alpha(H) \dot M. 
	 \] 
\end{defn}

\bigskip

It's still a bit fuzzy: we don't know what our linear functionals (eigenvalues) are and we also don't know what our $M$ (eigenvectors) look at. Thus let's spell everything out explicitly.

Take any  $H \in \mathfrak{h}$ and any $M \in \mathfrak{sl}_3\C$, namely: \[
		H = \def\a{\color{red}a} \begin{bmatrix} \a_1 & 0 & 0 \\ 0 & \a_2 &0 \\ 0 &0 &\a_3 \end{bmatrix}, M = \def\m{\color{blue}m} \begin{bmatrix} \m_{11} & \m_{12} & \m_{13} \\ \m_{21} & \m_{22} & \m_{23} \\ \m_{31} & \m_{32} & \m_{33} \end{bmatrix} 
	\]
Let's see what the adjoint $\operatorname{ad}_{H}(M)$ give us.
\begin{align*}
[H,M] &= HM - MH \\ &= \def\a{\color{red}a} \def\m{\color{blue}m} \begin{bmatrix} (\a_1-\a_1)\m_{11} & (\a_1-\a_2)\m_{12} & (\a_1-\a_3)\m_{13} \\ (\a_2-\a_1)\m_{21} & (\a_2-\a_2)\m_{22} & (\a_2-\a_3)\m_{23} \\ (\a_3-\a_1)\m_{31} & (\a_3-\a_2)\m_{32} & (\a_3-\a_3)\m_{33} \end{bmatrix} \\
		      &=  \def\a{\color{red}a} \def\m{\color{blue}m} \begin{bmatrix} 0 \cdot \m_{11} & (\a_1-\a_2)\m_{12} & (\a_1-\a_3)\m_{13} \\ (\a_2-\a_1)\m_{21} & 0 \cdot \m_{22} & (\a_2-\a_3)\m_{23} \\ (\a_3-\a_1)\m_{31} & (\a_3-\a_2)\m_{32} & 0 \cdot \m_{33} \end{bmatrix} 
\end{align*}

An immediate observation is that all diagnoal matrices are eigenvectors of $\mathfrak{h}$ with eigenvalue  $0$. In other words,  $\mathfrak{h}$ is an eigenspace  of itself with eigenvalue $0$.

(If one wants to be more rigorous, by eigenvalue  $0$, we mean a linear functional that sends everything to $0$, so it's a  $T: \mathfrak{h}\to \C$ s.t. $H \mapsto 0$. )

Another important observation is that, we could only have $6$ eigenvalues for each  $H \in \mathfrak{h}$: $a_1-a_2, a_1-a_3, a_2-a_3, a_2-a_1, a_3-a_1, a_3-a_2$. Because we want our eigenvalue to be applicable for all $H \in \mathfrak{h}$, whose only constraint is $a_1+a_2+a_3 = 0$, we could only afford to hold them separately as eigenvalue.

Let's them formally define our linear functional: 
\begin{defn}[$L_i$]
	Let $L_i: \mathfrak{h} \to \C$ be a linear functional s.t. \[
		L_i \left(\begin{bmatrix} a_1 & 0 &0 \\ 0 & a_2 & 0 \\ 0 &0 &a_3 \end{bmatrix}\right) = a_i
	\] it satisfies the requirement of a linear functional:

	$L_i(A+B) = L_i(A) + L_i(B), L_i(aA) = aL_i(A)$.
\end{defn}

\section{Weyl group}

\begin{defn}[adjoint map]	
	Let $G$ be a matrix Lie group, with Lie algebra  $\mathfrak{g}$. Then for each  $A \in G$, define a linear map $Ad_A: \mathfrak{g} \to  \mathfrak{g}$ by the formula \[
		Ad_A(X) = AXA^{-1}.
	\] 
\end{defn}

\begin{defn}[Weyl group]
	Let $\mathfrak{h}$ be the two-dimensional subspace of  $sl(3, \C)$. Let  $N$ be the subgroup of  $SU(3)$ consisting of those  $ A \in SU(3)$ s.t. $Ad_A(H)$ is an element of  $\mathfrak{h}$ for all  $H \in \mathfrak{h}$. Let $Z$ be the subgroup of  $SU(3)$ consisting of those  $A \in SU(3)$ s.t. $Ad_A(H)=H$ for all  $H \in \mathfrak{h}$.

	The \textbf{Weyl group} of $SU(3)$, denoted  $W$, is the quotent group  $N/Z$. 
\end{defn}



\nocite{*}

\newpage
\bibliography{ref}
\end{document}










