\documentclass[a4paper]{article}

\usepackage[utf8]{inputenc}
\usepackage{amssymb, amsfonts, latexsym, amsthm, amsmath, framed}
\usepackage{esvect}
\usepackage{parskip}

\usepackage{amsmath, amssymb, framed, tcolorbox}
\tcbuselibrary{theorems}

\usepackage{graphicx}

\usepackage{mathrsfs}
\usepackage[hidelinks,colorlinks=true,linkcolor=blue,citecolor=blue]{hyperref}

\usepackage{xcolor}

\usepackage{natbib}
\bibliographystyle{abbrvnat}
\setcitestyle{authoryear,open={},close={}}

\newcommand{\ba}{\backslash}
\newcommand{\Q}{\mathbb{Q}}
\newcommand{\C}{\mathbb{C}}
\newcommand{\R}{\mathbb{R}}
\newcommand{\N}{\mathbb{N}}
\newcommand{\Z}{\mathbb{Z}}
\newcommand{\F}{\mathbb{F}}
\newcommand{\rank}{\text{rank}}

% figure support
\usepackage{import}
\usepackage{xifthen}
\pdfminorversion=7
\usepackage{pdfpages}
\usepackage{transparent}
\newcommand{\incfig}[1]{%
	\def\svgwidth{\columnwidth}
	\import{./figures/}{#1.pdf_tex}
}

\newtheoremstyle{bfnote}%
  {}{}
  {}{}
  {\bfseries}{.}
  { }{\thmname{#1}\thmnumber{ #2}\thmnote{ (#3)}}

\theoremstyle{theorem} % set the style for the following theorems

\newtheorem{thm}{Theorem}[section] %\newtheorem{name}{display-text}[numbered-within]

\newtheorem{lem}[thm]{Lemma} %\newtheorem{name}[numbered-like]{display-text}
\newtheorem{cor}[thm]{Corollary}
\newtheorem{prop}[thm]{Proposition}
\newtheorem{alg}[thm]{Algorithm}

\theoremstyle{bfnote}                  % switch to a different style
\newtheorem{defn}[thm]{Definition}
\newtheorem{conj}[thm]{Conjecture}


\theoremstyle{example}                       % another style
\newtheorem{prob}[thm]{Problem}


\theoremstyle{remark}                       % another style
\newtheorem{exmp}[thm]{Example}  % (note:the "example" style is not really good for long examples-- typesets them in italics!)
\newtheorem{rem}[thm]{Remark}
\newtheorem{claim}[thm]{Claim}
\renewcommand{\theclaim}{}

%% If you use numbered equations in a long document, it is preferred to number
% as (x.y), where x is section number, y is equation number

\numberwithin{equation}{section}




\pdfsuppresswarningpagegroup=1

\usepackage{tcolorbox}
\usepackage{tikz}
\tcbuselibrary{theorems, skins, breakable}

\title{Algebra 2 Draft}
\author{Yuxuan Sun}


\begin{document}

\maketitle

\section{Introduction}
This paper aims to introduce the properties of irreducible representations of the symmetric group, an extension of the properties of irreducible representation in a boarder context: topological context, so that hopefully some geometric intuitions could be grasped. Besides abstract algebra, some knowledge about point-set topology is assumed.

The goal is to glance at the infinite-dimensional representations by studying Verma modules.

\section{Lie Algebra}


\begin{defn}[bracket, commutator, and Lie algebra, \citep{Humph}]\label{defn:bracket}
	A vector space $L$ over a field $F$, with an operation $L \times L \to  L$, denoted $(x,y) \mapsto [xy]$ and called the  \textbf{bracket} or \textbf{commutator} of $x$ and  $y$, is called a  \textbf{Lie algebra} over $F$ if the following axioms are satisfied:

	 \begin{enumerate}
		 \item The bracket operation is bilinear
\begin{enumerate}
	\item $[x,y_1+y_2] = [x,y_1] + [x,y_2]$ and $[x_1+x_2,y] = [x_1,y] + [x_2,y]$
	\item $[\lambda x, y] = \lambda[x,y] = [x, \lambda y]$
\end{enumerate}
		 \item $[x x] = 0$ for for all  $x \in L$
		 \item $[x[yz]]+[y[zx]] +[z[xy]] = 0$,  $x,y,z \in L$
	\end{enumerate}
	usually, $[x,y] = xy - yx$	
\end{defn}

Notice that  \textbf{1.} and  \textbf{2.} together implies the \textbf{anti-commutativity} of Lie-algebra, namely \[
	[x+y, x+y] = [x, x+y] + [y, x+y] = [x,x]+[x,y] + [y,x]+[y,y] = [x,y] + [y,x] = 0
\]  so we have another version of \textbf{2.} \[
[x,y] = -[y,x]
\]  

\bigskip

Let's consider our Lie group to be $SL_2(\C)$. Intuitively, the Lie algebra that corresponds to a given Lie group is the tangent space of the manifold at the identity element of the group.

To find the Lie algebra, $\mathfrak{sl}_2(\C)$, of Lie group ,$SL_2(\C)$, we need to use  $\epsilon$ to find the tangent space. Denote $\epsilon$ as a first order infinitesimal, that is, it's closer to  $0$ than any other real number, $\epsilon \neq 0$ and $\epsilon^2 = 0$. Thus the tangent space at the identity is simply all matrices $A$ s.t.  $I+A\epsilon \in SL_2(\C)$.

\[
	I + A \epsilon = \begin{bmatrix} 1 & 0 \\ 0 & 1 \end{bmatrix} + \begin{bmatrix} a & b \\ c & d \end{bmatrix} \epsilon = \begin{bmatrix} 1+a \epsilon & b \epsilon \\ c \epsilon & 1+d \epsilon \end{bmatrix}  
\]
Since we want $I + A \epsilon \in SL_2(\C)$, we need to calculate its determinant: \[
	\text{det}(I + A \epsilon) = (1+a \epsilon)(1+d\epsilon) - c b \epsilon ^2 = (1 + (a+d)\epsilon + a d \epsilon^2) - b c \epsilon^2 = 1 + (a+d) \epsilon 
\]
Thus for the determinant to be zero, we need the trace of $\mathfrak{sl}_2(\C)$ to be  $0$.

\bigskip

\begin{exmp}[basis of $\mathfrak{sl}_2\C$, \citep{fulton_representation_2004}]
	The trace of $\mathfrak{sl}_2\C$ needs to be $0$, and since they are $2 \times 2$  matrices, it means the number on the diagnoal must be the inverse of each other under addition. Explicitly, we could write $\mathfrak{sl}_2\C$ as:  \[
		\mathfrak{sl}_2\C = \left\{ \begin{bmatrix} a & b \\
		c & -a\end{bmatrix} \mid a,b,c \in \C  \right\}. 
			\] We could see that each matrix in $\mathfrak{sl}_2\C$ is determined by  $3$ numbers ( $a,b,c$), thus the basis has  $3$ elements. If we just let  $a,b,c$ to be  $1$ we have an intuitive basis composed by:  \[
				H = \begin{bmatrix} 1 & 0 \\ 0 & -1 \end{bmatrix}, \quad X = \begin{bmatrix} 0 & 1 \\ 0 & 0 \end{bmatrix}, \quad Y = \begin{bmatrix} 0 & 0 \\ 1 & 0 \end{bmatrix}  
			\]  
\end{exmp}

\begin{exmp}[bracket of $\mathfrak{sl}_2\C$]
	Let's define the bracket/commutator of $\mathfrak{sl}_2\C$ as the following. Given any  $A, B \in \mathfrak{sl}_2\C$, let: \[
	\left[ A, B \right] = AB - BA. 
\] One could check this bracket is well-defined. i.e. it satisfies the property we defined in \ref{defn:bracket} 
\end{exmp}

\section{Weyl group}

\begin{defn}[adjoint map]	
	Let $G$ be a matrix Lie group, with Lie algebra  $\mathfrak{g}$. Then for each  $A \in G$, define a linear map $Ad_A: \mathfrak{g} \to  \mathfrak{g}$ by the formula \[
		Ad_A(X) = AXA^{-1}.
	\] 
\end{defn}

\begin{defn}[Weyl group]
	Let $\mathfrak{h}$ be the two-dimensional subspace of  $sl(3, \C)$. Let  $N$ be the subgroup of  $SU(3)$ consisting of those  $ A \in SU(3)$ s.t. $Ad_A(H)$ is an element of  $\mathfrak{h}$ for all  $H \in \mathfrak{h}$. Let $Z$ be the subgroup of  $SU(3)$ consisting of those  $A \in SU(3)$ s.t. $Ad_A(H)=H$ for all  $H \in \mathfrak{h}$.

	The \textbf{Weyl group} of $SU(3)$, denoted  $W$, is the quotent group  $N/Z$. 
\end{defn}



\nocite{*}

\newpage
\bibliography{ref}
\end{document}










