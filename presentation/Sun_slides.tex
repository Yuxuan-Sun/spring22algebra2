\documentclass{beamer}

\usepackage[utf8]{inputenc}
\usepackage{amssymb, amsfonts, latexsym, amsthm, amsmath, framed}
\usepackage{esvect}
\usepackage{parskip}

\usepackage{amsmath, amssymb, framed, tcolorbox}
\tcbuselibrary{theorems}

\usepackage{mathrsfs}
% \usepackage[hidelinks,colorlinks=true,linkcolor=blue,citecolor=blue]{hyperref}

\usepackage{xcolor}

\usepackage{natbib}
\bibliographystyle{plainnat}

\newcommand{\ba}{\backslash}
\newcommand{\Q}{\mathbb{Q}}
\newcommand{\C}{\mathbb{C}}
\newcommand{\R}{\mathbb{R}}
\newcommand{\N}{\mathbb{N}}
\newcommand{\Z}{\mathbb{Z}}
\newcommand{\F}{\mathbb{F}}
\newcommand{\rank}{\text{rank}}

\newcounter{mytheorem}[section] \def\themytheorem{\thesection.\arabic{mytheorem}}

\definecolor{myteal}{cmyk}{0.5,0,0.15,0}
\usecolortheme[named=myteal]{structure}

\definecolor{my-yellow}{cmyk}{0,0.2,0.7,0,1.00}
\definecolor{my-blue}{cmyk}{0.80, 0.13, 0.14, 0.04, 1.00}
\definecolor{my-green}{cmyk}{0.4,0,0.4,0,1.00}
\tcbset{
defstyle/.style={fonttitle=\bfseries\upshape, colback=my-yellow!5,colframe=my-yellow!80!black},
theostyle/.style={fonttitle=\bfseries\upshape, colback=my-blue!5,colframe=my-blue!80!black},
corstyle/.style={fonttitle=\bfseries\upshape, colback=my-green!5,colframe=my-green!80!black},
}

\tcbmaketheorem{defn}{Definition}{theostyle}{mytheorem}{def}
\tcbmaketheorem{theom}{Theorem}{theostyle}{mytheorem}{theo}
\tcbmaketheorem{coro}{Corollary}{corstyle}{mytheorem}{cor}
\tcbmaketheorem{exm}{Example}{corstyle}{mytheorem}{eg}


\usetheme{Madrid}


\title{Lie algebra $\mathfrak{sl}_3$ and representation}
\subtitle{eigenspaces and representation decomposition for $\mathfrak{sl}_3$}
\date{Apr 25, 2022}
\author{Yuxuan Sun}
\institute{Haverford College}

\begin{document}
	
\begin{frame}
\titlepage
\end{frame}

\section{first}




\begin{frame}{Lie algebra}

\begin{exm}{$\mathfrak{sl}_3(\R)$}{}
	$\mathfrak{sl}_{3}$ is a vector space of $3 \times 3$ matrices with trace 0,in other words:  \[
		\mathfrak{sl}_{3} = \left\{ \begin{bmatrix} a & c & d \\
		e & -a+b & f \\ g & h & -b\end{bmatrix} \mid a,b,c,d,e,f,g,h \in \R \right\} 
			\] equppied with $[A,B] = AB - BA$
\end{exm}

\end{frame}


\addtocounter{mytheorem}{-1}
\begin{frame}{Lie algebra}

\begin{exm}{$\mathfrak{sl}_3(\R)$}{}
	$\mathfrak{sl}_{3}$ is a set of $3 \times 3$ matrices with trace 0, in other words:  \[
		\mathfrak{sl}_{3}(\R) = \left\{ \begin{bmatrix} a & c & d \\
		e & -a+b & f \\ g & h & -b\end{bmatrix} \mid a,b,c,d,e,f,g,h \in \R \right\} 
	\] 
\tcblower
the basis is: \[
	\mathcal{B}_{\mathfrak{sl}_{3}} = \left\{ H_1 = \begin{bmatrix} 1 & 0 &0 \\ 0 & -1 & 0 \\ 0 &0 &0 \end{bmatrix}, H_2 = \begin{bmatrix} 0 &0 &0 \\ 0 &1&0 \\ 0&0&-1 \end{bmatrix}   \right\} \bigcup \left\{ E_{i,j} \mid i \neq j\right\}. 
\] $E_{i,j}$ is a matrix where ${E_{i,j}}_{ij} = 1$ and other places are $0$.
\end{exm}

\end{frame}

\begin{frame}{representation and action}
	\begin{defn}{adjoint representation of $\mathfrak{sl}_3(\R)$}{}
		Give $X \in \mathfrak{sl}_3(\R)$, define a linear map $ad_X : \mathfrak{sl}_3(\R) \to \mathfrak{sl}_3(\R)$ as: \[
			ad_X(Y) = [X,Y] = XY - YX
		\] the map $X \mapsto ad_X$ is called an adjoint representation of $\mathfrak{sl}_3(\R)$.
	\end{defn} 
\end{frame}

\begin{frame}{some linear algebra}
	\begin{defn}{subspace $\mathfrak{h} \subseteq \mathfrak{sl}_3(\R)$}{}
		Let  $\mathfrak{h}$ be the two-dimensional subspace of all diagonal matrices, namely:  \[
			\mathfrak{h} = \text{span}{\left\{ H_1 = \begin{bmatrix} 1 & 0 &0 \\ 0 & -1 & 0 \\ 0 &0 &0 \end{bmatrix}, H_2 = \begin{bmatrix} 0 &0 &0 \\ 0 &1&0 \\ 0&0&-1 \end{bmatrix}   \right\} }
		\] 
	\end{defn}
\end{frame}


\addtocounter{mytheorem}{-1}
\begin{frame}{some linear algebra}
	\begin{defn}{subspace $\mathfrak{h} \subseteq \mathfrak{sl}_3(\R)$}{}
		Let  $\mathfrak{h}$ be the two-dimensional subspace of all diagonal matrices, namely:  \[
			\mathfrak{h} = \text{span}{\left\{ H_1 = \begin{bmatrix} 1 & 0 &0 \\ 0 & -1 & 0 \\ 0 &0 &0 \end{bmatrix}, H_2 = \begin{bmatrix} 0 &0 &0 \\ 0 &1&0 \\ 0&0&-1 \end{bmatrix}   \right\} }
		\] 
	\end{defn}
	What does it mean to be an eigenspace for $\mathfrak{h}$?

	For all $H \in \mathfrak{h}$, an eigenspace $\mathfrak{g}_{\lambda}$ has all $M \in \mathfrak{sl}_{3}(\R)$ s.t. $ad_{H}(M) = [H,M] = HM-MH = \lambda M?$
\end{frame}


\begin{frame}{some linear algebra}
	Let's pick an arbitrary $H \in \mathfrak{h}$ and $M \in \mathfrak{sl}_3(\R)$, namely: \[
		H = \def\a{\color{red}a} \begin{bmatrix} \a_1 & 0 & 0 \\ 0 & \a_2 &0 \\ 0 &0 &\a_3 \end{bmatrix}, M = \def\m{\color{blue}m} \begin{bmatrix} \m_{11} & \m_{12} & \m_{13} \\ \m_{21} & \m_{22} & \m_{23} \\ \m_{31} & \m_{32} & \m_{33} \end{bmatrix} 
	\] 
\end{frame}




\begin{frame}{some linear algebra}
	Let's pick an arbitrary $H \in \mathfrak{h}$ and $M \in \mathfrak{sl}_3(\R)$, namely: \[
		H = \def\a{\color{red}a} \begin{bmatrix} \a_1 & 0 & 0 \\ 0 & \a_2 &0 \\ 0 &0 &\a_3 \end{bmatrix}, M = \def\m{\color{blue}m} \begin{bmatrix} \m_{11} & \m_{12} & \m_{13} \\ \m_{21} & \m_{22} & \m_{23} \\ \m_{31} & \m_{32} & \m_{33} \end{bmatrix} 
	\]
	Let's see the result under adjoint action, i.e. $ad_{H}(M) = [H,M] = HM-MH = ?$
	\begin{align*}
		[H,M] &= \def\a{\color{red}a} \def\m{\color{blue}m} \begin{bmatrix} (\a_1-\a_1)\m_{11} & (\a_1-\a_2)\m_{12} & (\a_1-\a_3)\m_{13} \\ (\a_2-\a_1)\m_{21} & (\a_2-\a_2)\m_{22} & (\a_2-\a_3)\m_{23} \\ (\a_3-\a_1)\m_{31} & (\a_3-\a_2)\m_{32} & (\a_3-\a_3)\m_{33} \end{bmatrix} \\
		      &=  \def\a{\color{red}a} \def\m{\color{blue}m} \begin{bmatrix} 0 \cdot \m_{11} & (\a_1-\a_2)\m_{12} & (\a_1-\a_3)\m_{13} \\ (\a_2-\a_1)\m_{21} & 0 \cdot \m_{22} & (\a_2-\a_3)\m_{23} \\ (\a_3-\a_1)\m_{31} & (\a_3-\a_2)\m_{32} & 0 \cdot \m_{33} \end{bmatrix} 
	\end{align*}
\end{frame}

\begin{frame}{some linear algebra}
	\[ [H,M] = \def\a{\color{red}a} \def\m{\color{blue}m} \begin{bmatrix} 0& (\a_1-\a_2)\m_{12} & (\a_1-\a_3)\m_{13} \\ (\a_2-\a_1)\m_{21} & 0& (\a_2-\a_3)\m_{23} \\ (\a_3-\a_1)\m_{31} & (\a_3-\a_2)\m_{32} & 0 \end{bmatrix} \] 
	\def\a{\color{red}a}
	observation: recall our only constrain on $H$:  $a_1+a_2+a_3=0$, for $M$ to be an eigenvector under bracket  $H$, i.e.  $[H,M] = \lambda M$ for some  $\lambda$,  $M$ must be all  $0$ on the diagonal and only one place not equal to  $0$. 

%	For the second part, don't forget we have a constrain: $\a_1+\a_2+\a_3 = 0$ 

%	if, for example, $\lambda(\a_1-\a_3) = \lambda(\a_2-\a_3)$, we'll have $\a_1=\a_2$
\end{frame}


\begin{frame}{some linear algebra}
	\[ [H,M] = \def\a{\color{red}a} \def\m{\color{blue}m} \begin{bmatrix} 0& (\a_1-\a_2)\m_{12} & (\a_1-\a_3)\m_{13} \\ (\a_2-\a_1)\m_{21} & 0& (\a_2-\a_3)\m_{23} \\ (\a_3-\a_1)\m_{31} & (\a_3-\a_2)\m_{32} & 0 \end{bmatrix} \] 
	\def\a{\color{red}a}
	observation: recall our only constrain on $H$:  $a_1+a_2+a_3=0$, for $M$ to be an eigenvector under bracket  $H$, i.e.  $[H,M] = \lambda M$ for some  $\lambda$,  $M$ must be all  $0$ on the diagonal and only one place not equal to  $0$. 

the eigenspace is generated by the other basis element  $E_{i,j}$ where $i \neq j$ and $E_{i,j}$ has all but one entry $ij$ zero, for example:  \[
		E_{1,3} = \begin{bmatrix} 0 & 0 & 1 \\ 0 & 0 &0 \\ 0 &0 &0 \end{bmatrix} 
	\] 
\end{frame}

\begin{frame}{linear functional}	
	\[ [H,M] = \def\a{\color{red}a} \def\m{\color{blue}m} \begin{bmatrix} 0& (\a_1-\a_2)\m_{12} & (\a_1-\a_3)\m_{13} \\ (\a_2-\a_1)\m_{21} & 0& (\a_2-\a_3)\m_{23} \\ (\a_3-\a_1)\m_{31} & (\a_3-\a_2)\m_{32} & 0 \end{bmatrix} \]

	one could see we have potentially $6$ eigenvalues for each  $H \in \mathfrak{h}$: $a_1-a_2, a_1-a_3, a_2-a_3, a_2-a_1, a_3-a_1, a_3-a_2$. 
	% Since we are asking for the eigenspace of $\mathfrak{h}$, we need a way to get  $a_i-a_j$ from any  $H \in \mathfrak{h}$ for us, thus we could define a linear functional:

\end{frame}


\begin{frame}{linear functional}	
	Since we are asking for  eigenspaces of $\mathfrak{h}$, we need a way to get  $a_i-a_j$ from any  $H \in \mathfrak{h}$ for us, thus we could define a linear functional:

	\begin{defn}{$L_i$}{}
	$L_i$ is a linear functional s.t. \[
		L_i \begin{bmatrix} a_1 & 0 &0 \\ 0 & a_2 & 0 \\ 0 &0 &a_3 \end{bmatrix} = a_i 
	\] it satisfies the requirement of a linear functional: 

	$L_i(A+B) = L_i(A) + L_i(B), L_i(aA) = aL_i(A)$.
	\end{defn}
\end{frame}

\begin{frame}{eigenvalues and eigenspaces}
	Denote $\mathfrak{h}^*$ as the set of all eigenvalues (linear functionals) of  $\mathfrak{h}$ such that
	
	given the adjoint representation of  $\mathfrak{sl}_3(\R)$, the eigenspace, denoted $\mathfrak{g}_{\alpha}$, associated to the eigenvalue $\alpha \in \mathfrak{h}^*$ is the subspace of all  $M \in \mathfrak{sl}_3(\R)$ s.t. for all $H \in \mathfrak{h}$ \[
	H(M) = ad_H(M) = [H, M] = \alpha(H) \cdot M
\] 
\end{frame}


\section{representation}

\begin{frame}{why?}
	Notice that for the adjoint representation of $\mathfrak{sl}_3(\R)$, we have the following decomposition:  \[
		\mathfrak{sl}_3(\R) = \mathfrak{h} \oplus \left( \oplus \mathfrak{g}_\alpha \right) 
	\] where $\alpha$ ranges over $\mathfrak{h}^*$ and $\mathfrak{g}_{\alpha}$ is the eigenspace w.r.t $\alpha$.
\end{frame}

\begin{frame}{why?}
	Notice that for the adjoint representation of $\mathfrak{sl}_3(\R)$, we have the following decomposition:  \[
		\mathfrak{sl}_3(\R) = \mathfrak{h} \oplus \left( \oplus \mathfrak{g}_\alpha \right) 
	\] where $\alpha$ ranges over $\mathfrak{h}^*$ and $\mathfrak{g}_{\alpha}$ is the eigenspace w.r.t $\alpha$

	this is a special case of a general statement:
	\begin{theom}{}{}
		For any semisimple Lie algebra $\mathfrak{g}$ (it is a direct sum of simple Lie algebras (non-abelian Lie algebras without any non-zero proper ideals), we could find an abelian subalgebra(closed under bracket)  $\mathfrak{h} \subset \mathfrak{g}$ s.t. the action of $\mathfrak{h}$ on any representation $V$ could be written as a direct sum decomposition of $V$ into eigenspaces  $V_{\alpha}$ for $\mathfrak{h}$.
	\end{theom}
\end{frame}

\begin{frame}{graph of eigenspaces}
\begin{center} \includegraphics[scale=.3]{eigenvalue.jpg}\end{center}	
\end{frame}



\begin{frame}{jumping around ..}
	Given any $X \in  \mathfrak{g}_{\alpha}$ and $Y \in \mathfrak{g}_{\beta}$, where would $ad_{X}$ sends $Y$ to? i.e.  $ad_{X}(Y) = [X,Y] = ?$

	$\mathfrak{h}$ could help us, namely:
		\[[H,[X,Y]] = ((\alpha+\beta)(H)) \cdot [X,Y] \]

		in other words, $ad_{X}(Y)$ is an eigenvctor for $\mathfrak{h}$ with eigenvalue  $\alpha +\beta$.  \[
			ad_{\mathfrak{g}_{\alpha}}: \mathfrak{g}_{\beta} \to \mathfrak{g}_{\alpha+\beta}
		\] 
\end{frame}

\begin{frame}{jumping around}
	The action of $\mathfrak{g}_{L_1-L_3}$ is:
	\begin{center} \includegraphics[scale=.3]{L1-L3.png}\end{center}
\end{frame}

\begin{frame}{Bibliography}
	\nocite{*}
	\bibliography{ref}
\end{frame}

\end{document}


















