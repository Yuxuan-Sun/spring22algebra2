\documentclass[a4paper]{article}

\usepackage[utf8]{inputenc}
\usepackage{amssymb, amsfonts, latexsym, amsthm, amsmath, framed}
\usepackage{esvect}
\usepackage{parskip}

\usepackage{amsmath, amssymb, framed, tcolorbox}
\tcbuselibrary{theorems}

\usepackage{mathrsfs}
\usepackage[hidelinks,colorlinks=true,linkcolor=blue,citecolor=blue]{hyperref}

\usepackage{natbib}
\bibliographystyle{abbrvnat}
\setcitestyle{authoryear,open={},close={}}

\newcommand{\ba}{\backslash}
\newcommand{\Q}{\mathbb{Q}}
\newcommand{\C}{\mathbb{C}}
\newcommand{\R}{\mathbb{R}}
\newcommand{\N}{\mathbb{N}}
\newcommand{\Z}{\mathbb{Z}}
\newcommand{\F}{\mathbb{F}}
\newcommand{\rank}{\text{rank}}

% figure support
\usepackage{import}
\usepackage{xifthen}
\pdfminorversion=7
\usepackage{pdfpages}
\usepackage{transparent}
\newcommand{\incfig}[1]{%
	\def\svgwidth{\columnwidth}
	\import{./figures/}{#1.pdf_tex}
}
\newcounter{mytheorem}[section] \def\themytheorem{\thesection.\arabic{mytheorem}}
\tcbset{
defstyle/.style={fonttitle=\bfseries\upshape,
arc=0mm, colback=yellow!5,colframe=yellow!80!black}, 
theostyle/.style={fonttitle=\bfseries\upshape, colback=blue!5,colframe=blue!50!black},
corstyle/.style={fonttitle=\bfseries\upshape, colback=green!5,colframe=green!30!black},
}

\theoremstyle{definition}
\newtheorem{theorem}{Theorem}
\newtheorem{definition}{Definition}
\newtheorem{corollary}{Corollary}
\newtheorem{notation}{Notation}

\pdfsuppresswarningpagegroup=1

\title{Algebra 2 Term Paper Proposal}
\author{Yuxuan Sun}


\begin{document}
	
\maketitle

\section{Introduction}
This paper aims to offer sufficient introductions and examples to understand that given a finite solvable group $G$ which acts faithfully, irreducibly and quasi-primitively on a finite vector space $V$ and $G$ is not metacylic, $G$ always has a regular orbit on  $V$ except for a few small cases (\citep{yang_regular_2020}).


\section{Terms and Theorems}

The followings are from \citep{artinAlgebra}.

\begin{definition}[simple group]
	A group $G$ is \textit{simple} if it is not the trivial group and if it contains no proper normal subgroup - no normal subgroup other than  $\langle 1 \rangle $ and $G$.	
\end{definition}

\begin{corollary}
	Cyclic groups of prime order are simple groups.
\end{corollary}

\begin{definition}[solvable]
	A finite group $G$ is \textit{solvable} if it contains a chaine of subgroups \[
	G = H_0 \subset H_1 \subset H_1 \subset \ldots \subset  H_k = \left\{ 1 \right\}  
	\]  such that for every $i = 1, \ldots, k$, $H_i$ is a normal subgroup of $H_{i-1}$, and the quotient group $H_i\/H_{i+1}$ is a cyclic group.
\end{definition}

The followings are from \citep{fawcett_regular_2016}.

\begin{definition}[base]
	Let $G$ be a finite group acting faithfully on a set  $\Omega$. A \textit{base}  $\mathscr{B}$ for  $G$ is a non-empty subset of  $\Omega$ with the property that only the identity fixes every element of  $\mathscr{B}$.
\end{definition}

\begin{definition}[regular orbit]
	Let $G$ be a finite group acting faithfully on a set  $\Omega$. If a base $\mathscr{B} = \left\{ \omega \right\} $ for some $\omega \in \Omega$, the the orbit $\left\{ \omega g : g \in  G \right\} $ of $G$ on  $\Omega$ is regular. 
\end{definition}

The followings are from \citep{gelander_countable_2008}.

\begin{definition}{primitive action}
	An action of a group $G$ on a set  $X$ is \textit{primitive} if  $\left| X \right| > 1$ and there are no $G$-invariant equivalence relations on $X$ apart from the two trivial ones. 


	The trivial equivalence relations are those with a unique equivalence class, or with singletons as equivalence classes. When $\left| X \right| = 2$, we require that the action is not trivial.
\end{definition}

\begin{definition}{quasiprimitive action}
	An action is called \textit{quasiprimitive} if every normal subgroup acts either trivially or transitively.
\end{definition}

\begin{definition}{quasiprimitive group}
	A group is \textit{quasiprimitive} if it admits a faithful quasiprimitive action on a set.
\end{definition}

The following is from \citep{metacyclic}.

\begin{definition}{metacyclic}
	A group $G$ is \textit{metacyclic} if it has a cyclic normal subgroup  $N$ such that  $G \/ N$ is cyclic.
\end{definition}
	
The followings are notations from \citep{yang_regular_2020}.

\begin{notation}
	Let $G$ be a finite group, let  $S$ be a subset of  $G$ and let  $\pi$ be a set of different primes.

	For each prime $s$, we denote  \[
		SP_s(S) = \left\{ \langle x \rangle | o(x) = s,x \in S \right\} \quad \text{and} \quad EP_s(S) = \left\{ x | o(x) = s,x \in S \right\}  
	\] also, we denote \[
	SP(S) = \cup SP_s(S) \quad \text{and} \quad EP(S) = \cup EP_s(S)
	\]  \[
	EP_{\pi}(S) = \cup_{s \in \pi} EP_s(S)
	\] also, we denote \[
	NEP(S) = \left| EP(S) \right| \quad \text{and} \quad NEP_s(S) = \left| EP_s(S) \right| 
	\] \[
	NEP_{\pi}(S) = \left| EP_{\pi}(S) \right| 
	\] 
	
\end{notation}

\section{Outline}
\begin{enumerate}
	\item provide examples of: 
		\begin{itemize}
\item a finite solvable group $G$ acts faithfully on a finite vector space $V$ 
	\item a finite solvable group  $G$ acts faithfully, quasi-primitively on a finite vector space  $V$
	\item a metacyclic group $G$
\item a finite solvable, non-metacyclic, group $G$ acts faithfully on a finite vector space $V$ 
\item a finite solvable, non-metacyclic,  group $G$ acts faithfully , quasi-primitively on a finite vector space $V$
\item the notations listed above
\end{itemize}
\item Pick a theorem from the paper that could be expained regarding the limits of pages.
\end{enumerate}



\bibliography{ref}
\end{document}
